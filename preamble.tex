% Document info %%%%%%%%%%%%%%%%%%%%%%%%%%%%%%%%%%%%%%%%%%%%%%%%%%%%%%%%%%%%%%%
%\newcommand{\myTitle}{Fast solvers for elliptic partial differential equations \\ based on spectral and high-order methods}
\newcommand{\myTitle}{Fast Solvers for Elliptic Partial Differential Equations \\ Based on Spectral and High-Order Methods}
\newcommand{\myName}{Daniel Frank Fortunato}
\newcommand{\myDepartment}{School of Engineering and Applied Sciences}
\newcommand{\myUni}{Harvard University}
\newcommand{\myTime}{2020}

% Layout %%%%%%%%%%%%%%%%%%%%%%%%%%%%%%%%%%%%%%%%%%%%%%%%%%%%%%%%%%%%%%%%%%%%%%
\usepackage[%
    % drafting,%
    % minionpro,%
    % minionprospacing,%
    % eulerchapternumbers,%
    % pdfspacing,%
    % beramono,%
    % subfig,%
    % eulermath,%
    % listings,%
%    palatino=off,
	dottedtoc,%
    floatperchapter,%
%    parts,
%    linedheaders
]{classicthesis}
\usepackage{newpxtext}
\renewcommand{\oldstylenums}[1]{{\fontfamily{pplj}\selectfont #1}}

\usepackage[utf8]{inputenc}
\usepackage[T1]{fontenc}
\usepackage[english]{babel}
\usepackage{geometry}
\geometry{%
	left=3cm,
	right=3cm,
	top=2.5cm,
	bottom=2.5cm,
	paperwidth=8.5in,
	paperheight=11in,
%    bindingoffset=0.89cm,%
%    inner=2cm,%
%    outer=4cm,%
    marginparsep=0.6cm,%
    marginparwidth=3cm%
}
\usepackage{microtype}
\usepackage{hyperref}
\usepackage[dvipsnames]{xcolor}
\usepackage{enumitem}
\usepackage{pdfpages}
\usepackage[normalem]{ulem}
\usepackage{quoting}
%\usepackage{chngcntr}
\usepackage{xspace}
\usepackage{varwidth}
%\usepackage{fancyhdr}
\usepackage[title,titletoc]{appendix}
\usepackage{setspace}
\usepackage{lipsum}
\usepackage{etoolbox}
%\AtBeginEnvironment{quote}{\itshape}
\newcounter{savefootnote} % Save footnote counter
\usepackage{todonotes}

%% Try to keep footnotes on one page.
\interfootnotelinepenalty=10000

% Revision color
\colorlet{rev}{red}
\newcommand{\rev}[1]{\textcolor{rev}{#1}}

%% Add support for color in order to color the hyperlinks.
%\colorlet{exlinkcolor}{red!50!black}
\colorlet{exlinkcolor}{green!50!black}
%\AtBeginDocument{
%\hypersetup{
%	colorlinks=false,
%	hypertexnames=false,
%	breaklinks,
%	linkcolor=exlinkcolor,
%	linkbordercolor=exlinkcolor,
%	urlcolor=exlinkcolor,
%	citecolor=exlinkcolor,
%	%linktocpage = false,
%	linktoc=all
%}}
\urlstyle{same}
\newcommand{\email}[1]{\protect\href{mailto:#1}{#1}}
%\hypersetup{colorlinks=false, linktoc=all}
\hypersetup{
	colorlinks=true,
	hypertexnames=false,
	breaklinks,
	linkcolor=black,
	citecolor=exlinkcolor,
	urlcolor=exlinkcolor,
%	linkcolor=exlinkcolor,
%	linkbordercolor=exlinkcolor,
%	citebordercolor=exlinkcolor,
%   linktocpage = false,
	linktoc=all
}

% Custom section & TOC headers
\definecolor{chapcolor}{gray}{0.5}
\setcounter{secnumdepth}{3}
\makeatletter
	\titleformat{\chapter}[display]%
    	{\Huge}%
		%{\mbox{}\oldmarginpar{\vspace*{-3\baselineskip}\color{CTsemi}\chapterNumber\thechapter}}%
%		{\mbox{}\oldmarginpar{\vspace*{-3\baselineskip}\chapterNumber\thechapter}}%
%		{\raggedright\color{chapcolor}\fontsize{34}{34}\selectfont\spacedlowsmallcaps{Chapter \oldstylenums{\thechapter}}}%
		{\raggedleft\color{chapcolor}\fontsize{34}{34}\textit{\chaptername\xspace \thechapter}}%
		{1.2cm}%
%		{\normalsize\titlerule\vspace*{.9\baselineskip}\normalfont\LARGE\textbf}%
		{\normalfont\LARGE\textbf}%
		[\normalsize\vspace*{.8\baselineskip}\titlerule]
	\titleformat{\section}%
		{\normalfont\Large}%
		{\textbf{\MakeTextLowercase{\thesection}}}%
		{1em}%
		{\textbf}
	\titleformat{\subsection}%
		{\normalfont\large}
		{\textbf{\MakeTextLowercase{\thesubsection}}}%
		{1em}%
		{\textbf}
	\titleformat{\subsubsection}%
		{\normalfont\normalsize}%
		{\textbf{\MakeTextLowercase{\thesubsubsection}}}%
		{1em}%
		{\textbf}
	\titleformat{\subparagraph}[runin]%
		{\normalfont\normalsize}%
		{\textbf{\MakeTextLowercase{\subparagraph}}}%
		{1em}%
		{\textbf}
    \renewcommand{\cftchappresnum}{\normalfont\textbf}%
    \renewcommand{\cftchapaftersnumb}{\normalfont\textbf}%
    \renewcommand{\cftchapfont}{\normalfont\bfseries}% titles in bold
    \renewcommand{\cftchappagefont}{\normalfont\bfseries}% page numbers in bold
    \setlength{\cftbeforechapskip}{1.5em}%
%    \setlength{\cftbeforesecskip}{.1em}%
%    \setlength{\cftbeforesubsecskip}{.1em}%
%    \setlength{\cftbeforesubsubsecskip}{.1em}%
    \setcounter{tocdepth}{3}
%    \renewcommand{\chaptermark}[1]{\markboth{\spacedlowsmallcaps{#1}}{\spacedlowsmallcaps{#1}}}
    \renewcommand{\chaptermark}[1]{\markboth{#1}{#1}}
%    \renewcommand{\sectionmark}[1]{\markright{\textsc{\thesection}\enspace\spacedlowsmallcaps{#1}}}
%	\renewcommand{\sectionmark}[1]{\markright{\thesection\enspace{#1}}}
    \renewcommand*\chaptermarkformat{\itshape\chaptername{} \thechapter.\;\;}
%	\let\MakeMarkcase\spacedlowsmallcaps
	\let\MakeMarkcase\normalfont
	\renewcommand{\headfont}{\small}
	\renewcommand{\footfont}{\small}
	\automark[chapter]{chapter}
	\ihead{\MakeTextUppercase{\headmark}}
	\chead[]{}
	\ohead[]{} %\ohead{\pagemark}
	\ifoot[]{}
	\cfoot[\pagemark]{\pagemark}% only for plain.scrheadings page style (first page of a chapter)
	\ofoot[]{} %\ofoot[\pagemark]{}
	\deffootnote[1cm]{0em}{0em}{\textsuperscript{\thefootnotemark}}
\makeatother

\newpairofpagestyles{blankNumbered}{%
    \clearscrheadfoot%
    \clearpairofpagestyles%
    \KOMAoptions{headsepline=0pt}%
    \cfoot[\pagemark]{\pagemark}
    %\ofoot[\pagemark]{\pagemark}
}

\AtBeginEnvironment{appendices}{\crefalias{chapter}{appendix}}

%\makeatletter
%\g@addto@macro\appendix{%
%  \addtocontents{toc}{%
%    \protect\renewcommand{\protect\cftchappresnum}{\appendixname\space}%
%  }%
%}
%\makeatother

% Math %%%%%%%%%%%%%%%%%%%%%%%%%%%%%%%%%%%%%%%%%%%%%%%%%%%%%%%%%%%%%%%%%%%%%%%%
\usepackage{amsthm}
\usepackage{amsmath}
\usepackage{amsfonts}
\usepackage{amssymb}
\usepackage{amsopn}
\usepackage{bm}
\usepackage{mathrsfs}
\usepackage{mathdots}
\usepackage{mathtools}
\usepackage[binary-units=true]{siunitx}
\usepackage{stmaryrd}
\usepackage{dsfont}

\newtheorem{theorem}{Theorem}[section]
\newtheorem{lemma}[theorem]{Lemma}
%\newcommand{\newsiamthm}[2]{
%  \theoremstyle{plain}
%  \theoremheaderfont{\normalfont\sc}
%  \theorembodyfont{\normalfont\itshape}
%  \theoremseparator{.}
%  \theoremsymbol{}
%  \newtheorem{#1}[theorem]{#2}
%}
%
%% Other predefined theorem-like environments
%\newsiamthm{lemma}{Lemma}
%\newsiamthm{corollary}{Corollary}
%\newsiamthm{proposition}{Proposition}
%\newsiamthm{definition}{Definition}

%\theoremstyle{definition}
%\newtheorem{definition}[theorem]{Definition}
%\newtheorem{example}[theorem]{Example}
%\newtheorem{xca}[theorem]{Exercise}

%\theoremstyle{remark}
%\newtheorem{remark}[theorem]{Remark}

% References %%%%%%%%%%%%%%%%%%%%%%%%%%%%%%%%%%%%%%%%%%%%%%%%%%%%%%%%%%%%%%%%%%
%\usepackage[norefs,nomsgs]{refcheck}

%\usepackage[%
%    style=authoryear,%
%%    dashed=false,%
%%    doi=false,%
%%    isbn=false,%
%%    hyperref=auto,%
%%    backref,%
%%    backrefstyle=three,%
%%    maxbibnames=99,%
%%    maxcitenames=3,
%]{biblatex}
%\bibliography{references.bib}

%\makeatletter
%\newif\ifBR@star
%\newcommand\nobibliography{\@ifstar{\BR@nobib{}}{\BR@nobib}}
%\newcommand\BR@nobib[1]{%
%  \ifx\relax#1\relax\global\BR@startrue\else\global\BR@starfalse\fi
%  \begingroup
%  \ifBR@star\@fileswfalse\fi
%  \renewenvironment{thebibliography}[1]{%
%    \usecounter{enumiv}%
%    \renewcommand\item[1][]{%
%          \ifx\relax####1\relax\stepcounter\@listctr\fi}%
%    \ifBR@star \newcommand\BR@bibitem[2][]{}\else
%    \let\BR@bibitem=\bibitem\fi
%    \let\bibitem=\BR@b@bibitem}{}%
%    \let\@biblabel\@gobble
%  \bibliography{#1}\endgroup
%  \ifx\relax#1\relax\global\BR@startrue\else\global\BR@starfalse\fi}
%\newcommand\BR@b@bibitem[2][]{\ifx\relax#1\relax\BR@bibitem{#2}%
%   \else \BR@bibitem[#1]{#2}\fi \BR@c@bibitem{#2}}
%\def\BR@c@bibitem#1 #2 \par{{\let\protect\@unexpandable@protect
%      \expandafter \gdef\csname BR@r@#1\@extra@b@citeb\endcsname
%      {\BR@nodot#2\relax.\relax\relax}}}
%\def\BR@nodot#1.\relax#2\relax{#1}
%%\newcommand\bibentry[1]{\nocite{#1}{\frenchspacing
%%     \hyper@natanchorstart{#1\@extra@b@citeb}%
%%     \@nameuse{BR@r@#1\@extra@b@citeb}\hyper@natanchorend}}
%\newcommand\bibentry[1]{\nocite{#1}{\frenchspacing
%     \@nameuse{BR@r@#1\@extra@b@citeb}}}
%\providecommand\@extra@b@citeb{}
%\providecommand\hyper@natanchorstart[1]{}
%\providecommand\hyper@natanchorend{}
%\AtBeginDocument{%
%  \providecommand*{\urlprefix}{URL }
%  \providecommand*{\url}[1]{\texttt{#1}}
%}
%\AtBeginDocument{\let\BR@bib\bibliography
%  \def\bibliography{\global\BR@starfalse\BR@bib}}
%\AtEndDocument{\ifBR@star\PackageWarningNoLine{bibentry}
%  {You have used \string\nobibliography* \MessageBreak
%   without a following \string\bibliography.\MessageBreak
%   You may not be able to run BibTeX}\fi}
%\AtBeginDocument{\@ifpackageloaded{url}
%  {\providecommand{\doi}{doi: \begingroup \urlstyle{rm}\Url}}
%  {\providecommand{\doi}[1]{doi: #1}}}
%\makeatother

\usepackage{cite}
\usepackage{bibentry}
\makeatletter
\renewcommand{\bibentry}[1]{\nocite{#1}{\frenchspacing\@nameuse{BR@r@#1\@extra@b@citeb}}}
\makeatother

\nobibliography*
\renewcommand{\bibname}{References}

% Customize naming conventions with cleveref:
\usepackage[capitalize,nameinlink]{cleveref}
\crefname{section}{section}{sections}
\crefname{subsection}{subsection}{subsections}
\Crefname{section}{Section}{Sections}
\Crefname{subsection}{Subsection}{Subsections}
\Crefname{figure}{Figure}{Figures}
% Don't say equation in front on an equation.
\crefformat{equation}{\textup{#2(#1)#3}}
\crefrangeformat{equation}{\textup{#3(#1)#4--#5(#2)#6}}
\crefmultiformat{equation}{\textup{#2(#1)#3}}{ and \textup{#2(#1)#3}}
{, \textup{#2(#1)#3}}{, and \textup{#2(#1)#3}}
\crefrangemultiformat{equation}{\textup{#3(#1)#4--#5(#2)#6}}%
{ and \textup{#3(#1)#4--#5(#2)#6}}{, \textup{#3(#1)#4--#5(#2)#6}}{, and \textup{#3(#1)#4--#5(#2)#6}}
% But spell it out at the beginning of a sentence.
\Crefformat{equation}{#2Equation~\textup{(#1)}#3}
\Crefrangeformat{equation}{Equations~\textup{#3(#1)#4--#5(#2)#6}}
\Crefmultiformat{equation}{Equations~\textup{#2(#1)#3}}{ and \textup{#2(#1)#3}}
{, \textup{#2(#1)#3}}{, and \textup{#2(#1)#3}}
\Crefrangemultiformat{equation}{Equations~\textup{#3(#1)#4--#5(#2)#6}}%
{ and \textup{#3(#1)#4--#5(#2)#6}}{, \textup{#3(#1)#4--#5(#2)#6}}{, and \textup{#3(#1)#4--#5(#2)#6}}
% Make number non-italic in any environment.
\crefdefaultlabelformat{#2\textup{#1}#3}

% Add a serial/Oxford comma in \cref by default.
\newcommand{\creflastconjunction}{, and~}
\newcommand{\crefrangeconjunction}{--}

\makeatletter
\DeclareOldFontCommand{\rm}{\normalfont\rmfamily}{\mathrm}
\DeclareOldFontCommand{\sf}{\normalfont\sffamily}{\mathsf}
\DeclareOldFontCommand{\tt}{\normalfont\ttfamily}{\mathtt}
\DeclareOldFontCommand{\bf}{\normalfont\bfseries}{\mathbf}
\DeclareOldFontCommand{\it}{\normalfont\itshape}{\mathit}
\DeclareOldFontCommand{\sl}{\normalfont\slshape}{\@nomath\sl}
\DeclareOldFontCommand{\sc}{\normalfont\scshape}{\@nomath\sc}
\makeatother

% Graphics, figures, tables %%%%%%%%%%%%%%%%%%%%%%%%%%%%%%%%%%%%%%%%%%%%%%%%%%%
\usepackage{import}
\usepackage{grffile}
\usepackage{graphicx}
\usepackage{overpic}
\usepackage{tikz}
\usetikzlibrary{shapes,arrows,plotmarks,positioning,calc}
\usepackage{pgfplots}
\pgfplotsset{compat=newest}
\usepgfplotslibrary{patchplots}
\usepackage{caption}
\captionsetup{format=plain,%
             % labelfont={small,sf},% not necessary since `font' applies to both label and text
             % labelformat=mycaption,
             font={footnotesize,it}}
\usepackage[caption=false]{subfig}
\usepackage{booktabs}
\usepackage{multirow}
\usepackage{tabularx}
\usepackage{array}

\makeatletter
\newcommand{\multiline}[1]{%
  \begin{tabularx}{\dimexpr\linewidth-\ALG@thistlm}[t]{@{}X@{}}
    #1
  \end{tabularx}
}
\makeatother

\setlength\abovecaptionskip{10pt}
\setlength\belowcaptionskip{10pt}
\def\@figtxt{figure}
\long\def\@makecaption#1#2{%
    \footnotesize
    \setlength{\parindent}{1.5pc}
  \ifx\@captype\@figtxt
    \vskip\abovecaptionskip
    \setbox\@tempboxa\hbox{{\normalfont\scshape #1}. {\normalfont\itshape #2}}%
    \ifdim \wd\@tempboxa >\hsize
      {\normalfont\scshape #1}. {\normalfont\itshape #2}\par
    \else
      \global\@minipagefalse
      \hb@xt@\hsize{\hfil\box\@tempboxa\hfil}%
    \fi
  \else
    \hbox to\hsize{\hfil{\normalfont\scshape #1}\hfil}%
    \setbox\@tempboxa\hbox{{\normalfont\itshape #2}}%
    \ifdim \wd\@tempboxa >\hsize
      {\normalfont\itshape #2}\par
    \else
     \global\@minipagefalse
      \hb@xt@\hsize{\hfil\box\@tempboxa\hfil}%
    \fi
    \vskip\belowcaptionskip
  \fi}

% Algorithms %%%%%%%%%%%%%%%%%%%%%%%%%%%%%%%%%%%%%%%%%%%%%%%%%%%%%%%%%%%%%%%%%%
\usepackage{algorithmicx}
\usepackage[noend]{algpseudocode}
\usepackage{algorithm}

\renewcommand{\algorithmicrequire}{\textbf{Input:}}
\renewcommand{\algorithmicensure}{\textbf{Output:}}

\makeatletter
\algrenewcommand\ALG@beginalgorithmic{\small}
\makeatother

\usepackage{listings}
\usepackage{matlab-prettifier}
%\lstdefinestyle{matlab-custom}{
%%	language=Matlab,
%	basicstyle=\footnotesize\mlttfamily,
%	keywordstyle=\footnotesize\mlttfamily,
%	emph={function, if, else, elseif, for, end},
%	emphstyle={\color{blue}}
%%	commentstyle=\itshape\color{purple!40!black},
%%	identifierstyle=\color{blue},
%%	stringstyle=\color{orange}
%}
%%\lstset{emph={function, if, else, elseif, for, end}, emphstyle={\color{blue}}}

% Numbering %%%%%%%%%%%%%%%%%%%%%%%%%%%%%%%%%%%%%%%%%%%%%%%%%%%%%%%%%%%%%%%%%%%

\numberwithin{equation}{section}
\numberwithin{theorem}{section}
\numberwithin{algorithm}{section}
%\numberwithin{figure}{section}
%\numberwithin{table}{section}

% Macros %%%%%%%%%%%%%%%%%%%%%%%%%%%%%%%%%%%%%%%%%%%%%%%%%%%%%%%%%%%%%%%%%%%%%%%
\DeclareMathOperator{\dn}{dn}
\DeclareMathOperator{\tr}{tr}
\DeclareMathOperator{\vecop}{vec}
\DeclareMathOperator*{\argmax}{arg\,max}
\DeclareMathOperator*{\argmin}{arg\,min}
\renewcommand{\vec}[1]{\bm{#1}}
\newcommand{\dt}{\Delta t}
\newcommand{\trans}{\top}
\newcommand{\ldbrace}{\{\!\!\{}
\newcommand{\rdbrace}{\}\!\!\}}
\newcommand{\ldbrack}{[\![}
\newcommand{\rdbrack}{]\!]}
\newcommand{\avg}[1]{\ldbrace #1 \rdbrace}
\newcommand{\jump}[1]{\ldbrack #1 \rdbrack}
\newcommand{\flux}[1]{\widehat{#1}}
\protected\def\f{f}
\protected\def\c{c}
\def\ultraSEM{\texttt{\upshape ultraSEM}\xspace}
\newcommand{\dtn}{\Sigma}
\newcommand{\mytt}[1]{{\mlttfamily #1}}
\DeclarePairedDelimiter\ceil{\lceil}{\rceil}
\DeclarePairedDelimiter\floor{\lfloor}{\rfloor}
\newcommand{\R}[0]{\mathbb{R}}
\newcommand{\C}[0]{\mathbb{C}}
\newcommand{\ttrank}{{\rm rank}^{\rm TT}}
\newcommand{\mlrank}{{\rm rank}^{\rm ML}}
\newcommand{\cprank}{{\rm rank}^{\rm CP}}
\newcommand{\rank}{{\rm rank}}
\newcommand{\lex}{<_{\rm lex}}
\newcommand{\lexeq}{\le_{\rm lex}}
\def\CPP{{C\nolinebreak[4]\hspace{-.05em}\raisebox{.4ex}{\tiny\bf ++}}}

\protected\def\squaresymbol{%
    \scalebox{0.8}{%
        \raisebox{0.08em}{%
            \color[HTML]{0060ad}$\blacksquare$%
        }%
    }%
    \hspace{0.2em}%
}

\protected\def\bulletsymbol{%
    \scalebox{1.5}{%
        \raisebox{-0.03em}{%
            \color[HTML]{003866}$\bullet$%
        }%
    }%
    \hspace{0.2em}%
}

\protected\def\trianglesymbol{%
    \scalebox{1.1}[0.9]{%
        \raisebox{0.1em}{%
            \color[HTML]{008cff}$\blacktriangle$%
        }%
    }%
    \hspace{0.2em}%
}
\protected\def\xsymbol{%
    \scalebox{0.9}{%
        \raisebox{0.08em}{%
            $\bm{\times}$%
        }%
    }%
    \hspace{0.2em}
}
\protected\def\diamondsymbol{%
    \raisebox{0.1em}{%
        \scalebox{0.7}{%
            \rotatebox[origin=c]{45}{%
                \color[HTML]{4dafff}$\blacksquare$%
            }%
        }%
    }%
    \hspace{0.2em}%
}

\protected\def\triangledownsymbol{%
    \scalebox{1.1}[0.9]{%
        \raisebox{0.1em}{%
            \color[HTML]{004680}$\blacktriangledown$%
        }%
    }%
    \hspace{0.2em}%
}

\protected\def\plussymbol{%
    \scalebox{0.9}{%
        \raisebox{0.08em}{%
            $\bm{+}$%
        }%
    }%
    \hspace{0.2em}%
}

\protected\def\circlesymbol{%
    \scalebox{0.7}{%
        \raisebox{0.15em}{%
            $\bigcirc$%
        }%
    }%
    \hspace{0.2em}%
}

\def\intcolor{cyan!60}
\def\extcolor{red!60}
\def\radius{0.09}

%% Align dashed line to corners to shapes
\usetikzlibrary{decorations}
\makeatletter
\def\pgf@dec@dashon{5pt}
\def\pgf@dec@dashoff{5pt}
\pgfkeys{/pgf/decoration/.cd,
  dash on/.store in=\pgf@dec@dashon,
  dash off/.store in=\pgf@dec@dashoff
}
\pgfdeclaredecoration{aligned dash}{start}{
\state{start}[width=0pt, next state=pre-corner,persistent precomputation={
\pgfextract@process\pgffirstpoint{\pgfpointdecoratedinputsegmentfirst}%
\pgfextract@process\pgfsecondpoint{\pgfpointdecoratedinputsegmentlast}%
\pgfmathsetlengthmacro\pgf@dec@dashon{\pgf@dec@dashon}%
\pgfmathsetlengthmacro\pgf@dec@dashoff{\pgf@dec@dashoff}%
\pgfmathsetlengthmacro\pgf@dec@halfdash{\pgf@dec@dashon/2}%
}]{}
\state{pre-corner}[width=\pgfdecoratedinputsegmentlength, next state=post-corner, persistent precomputation={
%
  \pgfmathparse{int(ceil((\pgfdecoratedinputsegmentlength-\pgf@dec@dashon-\pgf@dec@dashoff)/(\pgf@dec@dashon+\pgf@dec@dashoff)))}%
  \let\pgf@n=\pgfmathresult
  \pgfmathsetlengthmacro\pgf@b%
    {\pgfdecoratedinputsegmentlength/(\pgf@n+1)-\pgf@dec@dashon}%
  \ifdim\pgf@b<\pgf@dec@dashoff\relax%
    \pgfmathparse{int(\pgf@n-1)}\let\pgf@n=\pgfmathresult%
    \pgfmathsetlengthmacro\pgf@b%
      {\pgfdecoratedinputsegmentlength/(\pgf@n+1)-\pgf@dec@dashon}%
  \fi%
  \pgfmathsetlengthmacro\pgf@b{\pgf@b+\pgf@dec@dashon}%
}]{%
  \pgfmathloop
  \ifnum\pgfmathcounter>\pgf@n%
  \else%
    \pgfpathmoveto{\pgfpoint{\pgf@b*\pgfmathcounter-\pgf@dec@halfdash}{0pt}}%
    \pgfpathlineto{\pgfpoint{\pgf@b*\pgfmathcounter+\pgf@dec@halfdash}{0pt}}%
  \repeatpgfmathloop%
  \pgfpathmoveto%
    {\pgfpoint{\pgfdecoratedinputsegmentlength-\pgf@dec@halfdash}{0pt}}%
  \pgfpathlineto%
    {\pgfpointdecoratedinputsegmentlast}
}
\state{post-corner}[width=0pt, next state=pre-corner]{
   \pgfpathlineto{\pgfpoint{\pgf@dec@halfdash}{0pt}}%
}
\state{final}{
  \pgftransformreset%
  \pgfpathlineto{\pgfpointlineatdistance{\pgf@dec@halfdash}{\pgffirstpoint}{\pgfsecondpoint}}%
}
}
\tikzset{aligned dash/.style={
  decoration={aligned dash, #1}, decorate
}}
